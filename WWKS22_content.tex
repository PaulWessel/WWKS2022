% Wessel, P., Watts, A. B., Kim, S.-S., and Sandwell, D. T., 2022
%   Models for the Evolution of Seamount, Geophys. J. Int.
%

\section{Introduction}
Volcanic seamounts are present in all the world's oceans and are some of the most ubiquitous
landforms on Earth. \citet{M1964} defined seamounts as isolated underwater volcanic constructions
that exceeded 1,000 m in height from base to summit, but with improved mapping capabilities we now
recognize features as small as 50–100 m in height as proper seamounts \citep[e.g.,~][]{SC1990}.
Found in many tectonic settings, seamounts are generally understood to have been created near mid-ocean
spreading ridges, around the transform faults and fracture zones, in plate interiors over upwelling
mantle plumes, and in island-arc convergent settings
\citep[e.g.,~][]{SC2010}. Because of their ubiquity, they are important for many disciplines, such as
marine geology, oceanography, biology, ecology, and possibly, economic geology if considering future harvesting of
scarce mineral resources \citep[e.g.,~][]{WSK2010,WK2010}. Given the vastness of the oceans, multibeam bathymetric
mapping by research vessels have to date only identified a few thousand of them, making seamount
characterization one of the last major frontiers in the Ocean sciences. In contrast, the ever-improving
quality and coverage of the Earth's gravitational field by
satellite altimetry \citep[e.g.,~][]{Setal2021}, from which seafloor relief may be approximated
\citep{SS1997}, has resulting in numerous studies refining the global seamount population and adding several
tens of thousand seamounts to a growing data base \citep{CS1988,WL1997,W2001,KL2004,KW2011,G2022}. 

\begin{figure}[h]
\centering
\noindent \includegraphics[width=\textwidth]{pdf/WWKS22_Fig_smt.pdf}
\caption{Two contrasting seamount morphologies in the west-central Pacific Ocean. a) Circular guyot in the Magellan chain,
Western Pacific with a smaller parasitic cone extending to the southwest. b) Small circular and
elliptical seamounts in the Equatorial Pacific near the Clarion-Clipperton fracture zone.}
\label{WWKS22_Fig_sm}
\end{figure}

While altimetric studies of seamounts are capable of identifying the location and approximate radii
and heights of the larger ($> \sim 1.5$ km) seamounts, multibeam bathymetric coverage is preferred for investigating
the shapes of seamounts in more detail (Fig. \ref{WWKS22_Fig_sm}).  Pioneering studies into seafloor morphology
\citep{JMS1983,SJ1988,S1988} have shown that typical seamounts can be approximated well by truncated cones
with both flattening (defined as the ratio between summit and basal radii) and slope being prescribed
to nominal values around 0.2 and $15^\circ$, respectively, but there is considerable scatter. Furthermore,
these parameters reflect only a smaller subset of Pacific seamounts.
Thus, while it appears that many seamounts, in particular smaller ones, are well approximated by a truncated cone,
there are situations where other models are desirable.  For instance, \citet{G2022} extended the work of
\citet{S1988} by using a Gaussian seamount model to reassess the width (one sigma) to height ratio of 739
well-mapped seamounts, confirming a value of 2.4.  Moreover, in analytic explorations of plate flexure
beneath axisymmetric seamounts, models based on parabolic shapes have been used as these loads have analytical
flexural solutions \citep[e.g.,~][]{LN1980,W2001,KW2010}.  Even simpler expressions apply for disc loads
\citep{BS1969}, leading to approximations of general axisymmetric shapes by stacks of discs of various heights,
radii and even densities \citep[e.g.,~][]{HC1994}.  Thus, for comparisons and completeness we will discuss these synthetic
seamount shapes in some details: Gaussian, polynomial, conic and parabolic.  All of these shapes may be truncated
(i.e., to match the observed shapes of guyots) and have either circular or elliptical bases.  Because investigations
into the global extrusive magma budget and its spatial distribution require seamount volumes and basal areas \citep[e.g.,~][]{B1982},
we are interested in developing expressions for their volumes, seafloor area coverage, and average heights
as functions of important morphological parameters such as flatness, basal radius, height, and ellipticity.
Furthermore, seamounts serve as loads in isostatic studies and so it is important to define their internal
density structure, especially when comparing the calculated surfaces of flexure with seismic observations.

In the following, the relationship between the height of the seamount as a function of radial distance from
its center, $r$, will be denoted $h(r)$. Because there is often extraneous seafloor structure near the base
(such as abyssal hills, fracture zones, or mass-wasting products from large-scale sector collapses, slump blocks
and debris flow deposits) complicating the identification of a unique basal contour, and because there may be uncertainties
related to a regional-residual separation \citep[e.g.,~][]{W2016}, we are concerned with measuring areas and volumes
above a prescribed noise floor.  Depending on the chosen seamount geometry, we obtain separate equations for areas,
volumes, and mean heights.  Moreover, to simulate the evolution of seamounts we may need to evolve them over a
specified life span, and hence the increments in volume between each time step will be needed. Because growth
and collapse are integral parts of seamount evolution \citep[e.g.,~][]{HJ2020} we also present simple models to
simulate sectoral flank collapses.  Finally, we will consider a general model for axisymmetric density variations
and assign vertically-averaged density contrasts to each increment.

\section{General Seamount Models}
\label{sec:shape}
Below we consider both circular and elliptical seamounts.  Because the volumes of these are proportional
to their basal areas, it is convenient to introduce a representative radius so we do not have to give
separate equations for circular and elliptical seamounts.  While for circular seamounts the representative
radius is simply the basal radius $r_0$, for the elliptical case with semi-major and semi-minor axes
$a_0$ and $b_0$, respectively, the representative radius will be their geometric mean, $r_0 = \sqrt{a_0 b_0}$.
We will further assume that bathymetric ``noise'' (or gravimetric noise if modeling free-air or vertical gravity
gradient anomalies \citep[e.g.,~][]{WL1997,W2001,KW2011}) prevents a basal zero-contour from being uniquely identified,
forcing us to instead track a slightly higher contour level to escape the noise floor: we label that contour
$h_n$.  Finally many, but not all seamounts are wave-trimmed at sea level to become guyots and have
a flat top area of radius $r_f < r_0$.  Introducing the flattening $f = r_f/r_0$ we can derive general expressions
that consider these various complications.

\subsection{Conical Seamount}

\begin{figure}[h]
\centering
\includegraphics[width=9cm]{pdf/WWKS22_Fig_cone.pdf}
\caption{Crossection of a truncated conical seamount of nominal height $h_0$ and basal radius $r_0$, but subject
to a noise floor at $h_n$.  Truncation yields a top radius of $r_f$. Heavy line is $r(h)$, while $r_n$ is the
radius of the seamount if only detected above the noise floor, $h_n$.}
\label{WWKS22_Fig_cone}
\end{figure}
For a truncated cone (Fig.~\ref{WWKS22_Fig_cone}), the height depends on radius as
\begin{equation*}
h(r) = \left \{ \begin{array}{cl}
	h_0, &	r \leq f_f \\*[1ex]
	\displaystyle \frac{h_0 (r_0 - r)}{r_0 (1-f)}, & r_f < r \leq r_0, \\*[1ex]
	0, & r > r_0
\end{array} \right.
\end{equation*}
We are interested in the area $A$, volume $V$, and mean height $\bar{h}$ above the noise floor $h_n$.
Since the noise floor will always intersect the seamount flank (i.e., $r_n > r_f$ for all $h_n < h_0$),
the solution for the noise-floor radius becomes
\begin{equation*}
r_n = r_0 \left ( 1 - \frac{h_n}{h_0}(1 - f) \right ).
\end{equation*}
Thus, introducing $e_c = 1 -f$ the seafloor area inside the base contour $h_n$ is simply
\begin{equation}
A_c = \pi r_n^2 = \pi r_0^2 \left ( 1 - e_c\frac{h_n}{h_0} \right )^2,
\end{equation}
with volume
\begin{equation}
V_c = \frac{\pi}{3e_c}r_0^2h_0 \left [ e_c^3 \left (\frac{1}{e_c} - \frac{h_n}{h_0} \right )^3 - f^3 \right ].
\end{equation}
The mean height of this truncated seamount is thus
\begin{equation}
\bar{h}_c =  \frac{V_c}{A_c} = \frac{h_0}{3} \frac{\left (\frac{1}{e_c} - \frac{h_n}{h_0} \right )^3 - \frac{f}{e_c}^3}{\left (\frac{1}{e_c} - \frac{h_n}{h_0} \right)^2}.
\end{equation}
When the noise-floor may be ignored (i.e., $h_n \approx 0$), the expressions simplify to the volume of a frustum of a cone:
\begin{equation*}
A_c = \pi r_0^2,	\quad V_c = \frac{\pi}{3e_c}r_0^2h_0 \left (1 - f^3 \right ), \quad \bar{h}_c =  \frac{h_0}{3e_c}\left (1 - f^3 \right ),
\end{equation*}
and in the further absence of truncation ($f = 0$), we recover the familiar geometric results
\begin{equation*}
V_c = \frac{\pi}{3}r_0^2h_0, \quad \bar{h}_c =  \frac{h_0}{3}.
\end{equation*}

\subsection{Parabolic Seamount}

\begin{figure}[h]
\centering
\includegraphics[width=9cm]{pdf/WWKS22_Fig_parabola.pdf}
\caption{Crossection of a truncated parabolic seamount of nominal height $h_0$ and radius $r_0$, but
subject to a noise floor at $h_n$.  Truncation yields a top radius of $r_f$. Heavy line is $h(r)$.}
\label{WWKS22_Fig_parabola}
\end{figure}
\noindent
For the truncated parabolic seamount (Fig.~\ref{WWKS22_Fig_parabola}), the height depends on radius as
\begin{equation*}
h(r) = \left \{ \begin{array}{cl}
	h_0, &	r \leq r_f \\*[1ex]
	\displaystyle \frac{h_0}{1-f^2} \left [1 - \left (\frac{r}{r_0} \right )^2 \right ], & r_f < r \leq r_0, \\*[1ex]
	0, & r > r_0
\end{array} \right.
\end{equation*}
Again, the intersection of the noise floor and the parabolic seamount flank yields the noise-floor
radius
\begin{equation*}
r_n = r_0 \sqrt{1 - \frac{h_n}{h_0}\left (1 - f^2 \right )}.
\end{equation*}
In this case we introduce $e_p = 1 - f^2$, and now the area inside the base contour $h_n$ becomes
\begin{equation}
A_p = \pi r_n^2 = \pi r_0^2 \left ( 1 - e_p \frac{h_n}{h_0} \right ).
\end{equation}
Likewise, the volume of the frustum of the paraboloid is the difference between two paraboloids, yielding
\begin{equation}
V_p = \frac{\pi}{2}r_0^2h_0 \left [ e_p \left (\frac{1}{e_p} - \frac{h_n}{h_0} \right )^2 - f^2 \left (\frac{1}{e_p} - 1 \right ) \right ].
\end{equation}
The mean height of this truncated seamount is thus
\begin{equation}
\bar{h}_p =  h_0 \frac{ \left [ e_p \left ( \frac{1}{e_p} - \frac{h_n}{h_0} \right )^2 -  f^2 \left (\frac{1}{e_p} - 1 \right ) \right ]}{2 \left ( 1 - e_p \frac{h_n}{h_0} \right )}.
\end{equation}
In the special case of no noise-floor, we obtain the simpler expressions
\begin{equation*}
A_p = \pi r_0^2,	\quad V_p = \frac{\pi}{2} r_0^2 h_0 \left (1 + f^2 \right ), \quad \bar{h}_p =  \frac{h_0}{2}\left (1 + f^2 \right ),
\end{equation*}
and for the no-truncation scenario we find again the familiar geometric terms
\begin{equation*}
V_p = \frac{\pi}{2}r_0^2h_0, \quad \bar{h}_p =  \frac{h_0}{2}.
\end{equation*}

\subsection{Gaussian Seamount}

\begin{figure}[h]
\centering
\includegraphics[width=9cm]{pdf/WWKS22_Fig_Gaussian.pdf}
\caption{Crossection of a truncated Gaussian seamount of nominal height $h_0$ and radius
$r_0$, but subject to a noise floor at $h_n$.  Truncation yields a top radius of $r_f$.}
\label{WWKS22_Fig_Gaussian}
\end{figure}
\noindent
For an untruncated Gaussian seamount (dashed line in Fig. \ref{WWKS22_Fig_Gaussian}), we will parameterize it to have
a basal radius $r_0$ that equates to $3\sigma$; thus the height follows
\begin{equation*}
h(r) = h_0 e^{-\frac{9}{2} \left (\frac{r}{r_0} \right)^2}.
\end{equation*}
Unlike the previous expressions, the Gaussian does not ever reach zero for large radii, hence our decision to
truncate the expressions at $\pm3\sigma$ means we are ignoring a very small ($<1$\%) part of the seamount.
In the general case of a truncated Gaussian seamount, we need our $h(r)$ expression to have a similar
shape until we reach the truncated radius, $r_f$, where the height should be $h_0$. We see that
by evaluating $h(r_f)$ we obtain the required scaling term of $e^{\frac{9}{2}f^2}$.
Hence, our general expression for the height of a truncated Gaussian seamount (Fig.~\ref{WWKS22_Fig_Gaussian}) becomes
\begin{equation*}
h(r) = \left \{ \begin{array}{cl}
	h_0, &	r \leq r_f \\*[1ex]
	\displaystyle h_0 e^{\frac{9}{2} \left [ f^2 - \left (\frac{r}{r_0} \right )^2 \right ]}, & r_f < r \leq r_0, \\*[1ex]
	0, & r > r_0
\end{array} \right.
\end{equation*}
The intersection of the noise floor and the Gaussian seamount is obtained by equating $h(r_n) = h_n$,
which yields a noise-floor radius of
\begin{equation*}
r_n = r_0 \sqrt{f^2 - \frac{2}{9} \ln{\left (\frac{h_n}{h_0} \right )}}.
\end{equation*}
We can now evaluate the seafloor area inside the base contour $h_n$ to find
\begin{equation}
A_g = \frac{2\pi}{9} r_0^2 \left [ \frac{9}{2}f^2  - \ln{\left( \frac{h_n}{h_0} \right )} \right ].
\end{equation}
To obtain the volume of the seamount between its surface and the noise-floor plane at $h_n$ we
must integrate the expression for $h(r)$ from the perimeter of the truncated
surface to $r_n$, add the volume of the central cylindrical core and remove
the volume of the base pedestal below $h_n$:
\begin{equation*}
V_g = 2\pi \int_{r_f}^{r_n} h(r) r dr + \pi r_0^2 f^2 h_0 - \pi r_n^2 h_n.
\end{equation*}
Thus, the total volume of the Gaussian frustum becomes
\begin{equation}
V_g = \frac{2}{9} \pi r_0^2 h_0 \left \{ e_g - \frac{h_n}{h_0} \left [ e_g - \ln{\left (\frac{h_n}{h_0}\right )} \right] \right \},
\end{equation}
where we have introduced $e_g = 1 + \frac{9}{2}f^2$.
The mean seamount height can therefore be evaluated as
\begin{equation}
\bar{h}_g =  h_0 \frac{ e_g - \frac{h_n}{h_0} \left [ e_g - \ln{\left (\frac{h_n}{h_0}\right )} \right]}{\frac{9}{2} - \ln{\left( \frac{h_n}{h_0} \right )}}.
\end{equation}
As before, ignoring the noise-floor ($r_n \rightarrow \infty$) yields the simplifications
\begin{equation*}
V_g = \frac{2}{9} \pi r_0^2 h_0 e_g , \quad \bar{h}_g =  \frac{2}{9} h_0 e_g,
\end{equation*}
and with no truncation we reach the simplest terms
\begin{equation*}
V_g = \frac{2}{9} \pi r_0^2 h_0, \quad \bar{h}_g =  \frac{2}{9} h_0.
\end{equation*}

\subsection{Polynomial Seamount}

\begin{figure}[h]
\centering
\includegraphics[width=9cm]{pdf/WWKS22_Fig_polynomial.pdf}
\caption{Crossection of a truncated polynomial seamount of nominal height $h_0$ and radius
$r_0$, but subject to a noise floor at $h_n$.  Truncation yields a top radius of $r_f$. Heavy line is $h(r)$.}
\label{WWKS22_Fig_polynomial}
\end{figure}
One of the``unfortunate'' properties of the Gaussian is its slow asymptotic approach to zero. For this reason,
a polynomial model for a seamount shape that approximates the Gaussian curve but goes exactly to zero at the
given radius $r_0$ may be used \citep[e.g.,~][]{KW2011}.  The radial shape for a unit height untruncated seamount
of normalized radius $u = r/r_0$ is
\begin{equation*}
	p(u) = \frac{(1+u)^3(1-u)^3}{1+u^3}.
\end{equation*}
For a truncated polynomial seamount (Fig.~\ref{WWKS22_Fig_polynomial}) we must evaluate $p(f)$ and obtain the scaling
required to yield the nominal height $h_0$ for the general expression as
\begin{equation*}
h(r) = \left \{ \begin{array}{cl} 
	h_0, & r \leq r_f \\
	\frac{h_0}{p(f)} p(r/r_0), & r_f < r \leq r_n \\
	0, & r > r_0
\end{array} \right.
\end{equation*}
Unlike the previous seamount shapes, the polynomial model has no analytic solution for the noise-floor
radius satisfying $h(r_n) = h_n$.  Instead, we numerically solve this equality for $r_n$, which
trivially yields a seafloor area of $A_o = \pi r_n^2$.
As in the Gaussian case, we must integrate $p(u)$ over suitable limits to find the volume $V_1$.
The complete axisymmetric volume integral of $p(u)$ is analytical and was symbolically obtained
in MATLAB to be
\begin{equation*}
I_1 = 2\pi\int_0^1 p(u) udu = 2 \pi \left (\frac{\pi \sqrt{3}}{3} - \frac{17}{10} \right ).
\end{equation*}
For an upper limit $u < 1$ we still obtain an analytical expression via MATLAB, written as
\begin{equation*}
%2*pi*(3^(1/2)*atan(3^(1/2)*((2*f)/3 - 1/3)) - (3*log(f^2 - f + 1))/2 - 3*f + (pi*3^(1/2))/6 + f^2/2 + f^3 - f^5/5)
	I(u) = 2\pi \left [ \frac{\pi \sqrt{3}}{6} + \sqrt{3} \tan^{-1} \left( \frac{\sqrt{3}}{3}(2u - 1) \right ) - \frac{3}{2}\ln (u^2 - u + 1) - 3u + \frac{u^2}{2} + u^3 - \frac{u^5}{5} \right ].
\end{equation*}
To obtain the volume under the surface $h(r)$ above the noise floor $h_n$ requires similar gymnastics
as was done for the Gaussian case, yielding
\begin{equation}
	V_o = r_0^2 h_0 \left [ \frac{I(r_n/r_0) - I(f)}{p(f)} + \pi f^2 \right ] - \pi r_n^2h_n.
	\label{eq:poly_v2a}
\end{equation}
Here, we obtain the mean height $\bar{h}_o$ as the ratio between Eqs. (\ref{eq:poly_v2a}) and the area.
In the case where there is no noise floor ($h_n = 0$) but still flattening we similarly find
\begin{equation*}
	V_o = r_0^2 h_0 \left [ \frac{I_1 - I(f)}{p(f)} + \pi f^2 \right ],\quad\bar{h}_o = \frac{h_0}{\pi}\left [ \frac{I_1 - I(f)}{p(f)} + \pi f^2 \right ],
\end{equation*}
and for the non-truncation scenario it simplifies further to
\begin{equation*}
	V_o = r_0^2 h_0 I_1, \quad \bar{h}_o = \frac{h_0 I_1}{\pi}.
\end{equation*}

\begin{figure}[h]
\centering
\noindent\includegraphics[width=9cm]{pdf/WWKS22_Fig_disc.pdf}
\caption{Crossection of a circular disc of nominal height $h_0$ and radius
$r_0$, but subject to a noise floor at $h_n$. Truncation $f$ does not apply.}
\label{WWKS22_Fig_disc}
\end{figure}
While the definite integral $I_1$ is specific for the polynomial model
we can trivially derive comparable expressions for the other seamount shapes discussed earlier,
allowing a direct comparison between them, including a reference disc (Fig.~\ref{WWKS22_Fig_disc}). In the
case of neither a noise-floor nor truncation, Table 1 shows the various integrals which represent
the volumes of seamounts with unit radius and height.  Hence, scaling $I_1$ by $r_0^2 h_0$
gives actual seamount volumes.

\begin{table}[h]
\centering
\begin{tabular}{c c c c c} \hline
  \bf{Gaussian}  &                               \bf{Polynomial}                  &  \bf{Cone}  &    \bf{Parabola}    & \bf{Disc}  \\ \hline
$\frac{2\pi}{9}$ & $2 \pi\left (\frac{\pi \sqrt{3}}{3} - \frac{17}{10} \right )$ & $\frac{\pi}{3}$ & $\frac{\pi}{2}$ &   $\pi$    \\ 
      0.6981      &                               0.7150                           &      1.0472     &       1.5708    &   3.1415   \\ \hline
\end{tabular}
\caption{Table 1. Unit size volumes $I_1$ for different seamount shapes for no truncation ($f = 0$).}
\label{tbl:Ival}
\end{table}

Table 1 suggests that, for flexural studies, the shape of the approximation to a seamount may be important.
For a given load and hence load volume, different seamount shape models will require different basal radii
$r_0$ depending on $I_1$.  For instance, a parabolic approximation will require a seamount radius that is 2/3
of that required for a Gaussian seamount of same volume and height, or $\sim 3/4$ of that if a fixed slope rather
than height is to be maintained. For a fixed elastic thickness these load shapes will result in different deflections,
even though the flexural response is dominated by the first moment or the load \citep{M1981}.

\section{Evolution of Seamount Shapes}
\begin{figure}[h]
\centering
\includegraphics[width=9cm]{pdf/WWKS22_Fig_flux.pdf}
\caption{Cumulative volume flux is the fraction of total volume as a function of normalized time span.
We prefer a Gaussian model for flux, which yields a cumulative error-function curve (heavy line). In contrast,
a constant eruption rate yields the dashed linear curve.
The Gaussian curve is considered more appropriate based on observations of Hawaiian volcanoes \citep{DS1996}.}
\label{WWKS22_Fig_flux}
\end{figure}
Most flexural studies are concerned with loads emplaced a long time ago, making an elastic plate over an inviscid substrate
a valid rheological model \citep[e.g.,~][]{WC74}.  However, in areas of active volcanism more sophisticated rheologies that take into
account the viscoelastic response of the lithosphere to temporally varying loads may be required \citep[e.g.,~][]{ZW2013}.
In such cases, there will be a need to consider the evolving submarine loading of the plate. To simulate the construction
of a seamount over its life span we require a prescribed volume flux history (e.g., Fig.~\ref{WWKS22_Fig_flux})
and the assumption that the average slope of the seamount stays constant.  The latter requirement may not be
appropriate for islands since these typically exhibit different slopes during their submarine and subaerial
phases.  However, since unmapped seamounts obviously excludes islands the assumption is justified for this
work. The flux curve is usually assumed to be linear or Gaussian; the latter best mimics the voluminous shield
building phase as observed in the Hawaiian islands \citep{DS1996}.  The average slope constraint leads to
incremental layers of near-constant thickness that can be approximated by thin discs for simple analytical
flexure calculations (or used as is for numerical work). The normalized volume flux curves considered here are thus
\begin{equation*}
q(t) = \left \{ \begin{array}{cl}
	\frac{6}{\Delta t \sqrt{2 \pi}} e^{-18\left (\frac{t - t_m}{\Delta t} \right )^2}, &	\mbox{Gaussian} \\*[1ex]
	\frac{1}{\Delta t}, & \mbox{Linear}
\end{array} \right.,
\end{equation*}
where $\Delta t = t_1 – t_0$ is the life span of seamount construction and $t_m$ is the mid-point in time.
The normalized volume produced from the start of construction ($t_0$) to a time $t \leq t_1$ is thus the
integral of $q(t)$, i.e.,
\begin{equation}
v(t) = \int_{t_0}^t q(t')dt' = \left \{ \begin{array}{cl}
	\frac{1}{2} \left [ 1 + \mbox{erf}{\left (6\frac{t - t_m}{\sqrt{2}\Delta t} \right )} \right ], &	\mbox{Gaussian} \\*[1ex]
	\frac{t - t_0}{\Delta t}, & \mbox{Linear}
\end{array} \right ..
\label{eq:flux}
\end{equation}
Because for the Gaussian flux model, $v(t_0)$ is not quite 0 and $v(t_1)$ is not quite 1 we simply transform
and scale $v(t)$ with these values to make it so.
If we slice a truncated cone from the top and moving downward, the height of the remaining cone has constant
radius $r_0$ and slope but lower height, $h(t)$.  Given the fixity of the flank slope, this new height
should correspond to a reduced radius, $r(t)$.  Introducing
$\phi(t) = r(t)/r_0$ we may use the equation for the height of a cone and get
\begin{equation*}
h(t) = \frac{h_0 (r_0 - r(t))}{r_0 (1-f)} = h_0\frac{1 - \phi(t)}{1 - f}.
\end{equation*}
If we equate the sliced-off top volume with the volume fraction $v(t)$ of the total volume $V_0$,
then we can write the remaining volume as
\begin{equation*}
V_0 (1 - v(t)) = \frac{\pi r_0^2 h(t)}{3 (1-f)} (1 - \phi(t)^3) = K_c (1 - \phi(t)^3),
\end{equation*}
which we solve for $\phi(t)$, yielding
\begin{equation*}
\phi(t) = \left [1 - \frac{V_0 (1 - v(t))}{K_c} \right ]^{\frac{1}{3}},
\end{equation*}
from which we can obtain $h(t)$.
\begin{figure}[h]
\centering
\includegraphics[width=12cm]{pdf/WWKS22_Fig_increments.pdf}
\caption{Cumulative volume flux is the fraction of total volume as a function of normalized time span.
(left) Cumulative topographic relief over five consecutive and equally long time steps. (right) The incremental
changes for each of the time steps.}
\label{WWKS22_Fig_increments}
\end{figure}
If instead we are growing a parabolic seamount we start with the relationship
\begin{equation*}
h(t) = \frac{h_0}{1-f^2} \left [1 - \left (\frac{r(t)}{r_0} \right )^2 \right ] = h_0 \frac{1 - \phi(t)^2}{1 - f^2}.
\end{equation*}
Now our volume equation becomes
\begin{equation*}
V_0 (1 - v(t)) = \frac{\pi r_0^2 h(t)}{2 (1-f^2)} (1 - \phi(t)^4) = K_p (1 - \phi(t)^4).
\end{equation*}
Similarly, we solve this equation for the changing fraction and obtain
\begin{equation*}
\phi(t) = \left [1 - \frac{V_0 (1 - v(t))}{K_p} \right ]^{\frac{1}{4}}.
\end{equation*}
Next, if instead we are growing a Gaussian shape we start with
\begin{equation*}
h(t) = h_0 e^{\frac{9}{2} \left [ f^2 - \left (\frac{r(t)}{r_0} \right )^2 \right ]} = h_0 e^{\frac{9}{2}\left(f^2 - \phi(t)^2\right)}
\end{equation*}
Now our volume equation becomes
\begin{equation*}
V_0 (1 - v(t)) = \frac{2\pi r_0^2 h(t)e^{\frac{9}{2}f^2}}{9} \left (1 + \frac{9}{2}\phi(t)^2\right ) e^{\frac{9}{2}\phi(t)^2} = K_g (1 + \frac{9}{2}\phi(t)^2) e^{\frac{9}{2}\phi(t)^2}.
\end{equation*}
We solve this equation for the changing fraction by iterating over the initial guess
\begin{equation*}
\phi(t)_{i+1} = \frac{\sqrt{2}}{3} \left \{ - \ln \left [\frac{V_0 (1 - v(t))}{K_g \left (1 + \frac{9}{2}\phi(t)^2_i \right )} \right ] \right \}.
\end{equation*}
We replace the initial guess $\phi(t)_i$ with $\phi(t)_{i+1}$ and iterate until the difference
is less than some tolerance.

Finally, for a polynomial seamount we start with the relationship
\begin{equation*}
h(t) = h_0 \frac{p(\phi(t))}{p(f)},
\end{equation*}
and the balance of volumes leads to another implicit equation for $\phi(t)$ that must be solved numerically:
\begin{equation*}
	\pi \phi^2(t) p(\phi(t)) - I(\phi(t)) = (v(t)-1) \left [\pi f^2 p(f) - I(f) \right ] - v(t) I_1.
\end{equation*}
Given $\phi(t)$ for the desired seamount model and the choice of a flux curve from (\ref{eq:flux}) we
can simulate the growth of seamounts and separate out the exact incremental mass layers that are added as a
function of time.  Fig. \ref{WWKS22_Fig_increments} illustrates five life-stages for a Gaussian seamount following a
Gaussian cumulative flux curve.
For circular seamounts, it may be advantageous to make further simplifications.  Armed with the solutions
for $\phi(t)$, we may evaluate $h(t)$ for each time step and use the difference
in height between two times $t_1$ and $t_2$ to obtain $\Delta h(t)$.  From this information we solve for
the mean radius for the equivalent disc mass,
\begin{equation*}
\bar{r}(t) = \sqrt{\frac{V(t_2) - V(t_1)}{\pi \left( h(t_2) - h(t_1) \right)}} = \sqrt{\frac{\Delta V(t)}{\pi \Delta h(t)}},
\end{equation*}
which may be used to approximate the loading history (and the lithosphere's flexural response) to a circular seamount of interest
by taking advantage of the analytical solutions for disc flexure \citep{BS1969}.

\section{Sectoral Flank Collapse and Redistribution of Mass}

The literature demonstrates that seamounts experience a wide range of sizes of submarine landslides,
some involving mass-wasting of hundreds to thousands of cubic kilometers of material.  These collapses have been particularly well
documented for the Hawaiian \citep[e.g.,~][]{M1989,M1994}, Society \citep[e.g.,~][]{C2001}, Marquesas \citep[e.g.,~][]{F1994}
and Canary \citep[e.g.,~][]{WM1995,CJC1999} Islands, making it clear that construction and collapse are intertwined
throughout the early evolution of many volcanic islands \citep{HJ2017,HJ2020}. While there are many aspect of mass
redistribution that are of interest to scientists, one particular effect is the isostatic adjustment due to the
redistribution of the seamount load on the lithosphere following a landslide and implications for uplift, stresses and rift zone formation.
An early study by \citet{SW2000} approximated the shapes of the landslide scarps and used
these models to perform flexural calculations using the seamount load distribution before and after a landslide event.
Because the evolution from smaller circular seamounts through larger and more complex structures dominated by rift
zones \citep[e.g.,~][]{M2001} to eventual flank collapse appears to be the norm rather than the exception, we have developed
simple models to facilitate research into the temporal evolution of oceanic seamounts.

\subsection{A geometric model for a slide}

Figure~\ref{WWKS22_Fig_slide} illustrates the geometry of a sectoral landslide and three functions $q(r)$, $s(\alpha)$ and $\psi(t)$ whose
parameters control the radial and azimuthal shape of the slide surface and the evolution of the slide volume over time.
We chose to model the scalloped cross-sectional slide shape over the radial range $r_2$ to $r_1$ by
\begin{equation}
h_s(r) = h_1 + \Delta h \cdot q(u),
\end{equation}
where $\Delta h = h_2 - h_1$ is the escarpment height of the rupture and $q(u)$ is the normalized radial shape
of the slide
\begin{equation}
q(u) = u_0 \left (\frac{1 + u_0}{u + u_0} - 1 \right ),
\end{equation}
i.e., an hyperbolic curve.  Here, $u$ is the normalized distance from $r_2$ to $r_1$, given by
\begin{equation}
u = \frac{r-r_2}{r_1 - r_2} = \frac{r-r_2}{\Delta r},
\end{equation}
hence $r = r_2 + u \Delta r$. We may modify the parameter $u_0$ to carve more or less deeply
into the seamount core. Figure~\ref{WWKS22_Fig_slide}b shows typical shapes of $q(u)$ for some values of $u_0$.

Most slides do not involve the entire seamount, of course. Instead, they only affect a limited sector of it.
Thus, our slide sector may be specified by two azimuths $\alpha_1$ and $\alpha_2$ and the slide will only affect
that part of the seamount with a volume fraction $\theta = (\alpha_2 - \alpha_1)/360^{\circ}$.  More than one slide may
occur on any seamount.

\begin{figure}[H]
\centering
\includegraphics[width=9cm]{pdf/WWKS22_Fig_slide.pdf}
\caption{(a) Geometry for an \emph{ad hoc} landslide approximation.  The slide material (pink)
  will be deposited at the toe of the seamount (light blue) starting at a height of $h_c$ and linearly
  tapering to zero at a distal point $r_d$.
  (b) A variety of slide shapes are possible by varying the parameter $u_0$.  The slide area for a conical seamount
  would be the area between the flank (dashed line) and the selected curve. A smaller $u_0$ will cut more deeply
  into the seamount.
  (c) A range of azimuthal variation in slide height can be achieved by modulating the power parameter, $p$.
  This variation means the slide volume is reduced by $1 - \bar{s}$ (dashed lines). E.g., for $p = 2$ the slide
  volume is only 67\% of the volume we would have if there was no azimuthal variation (i.e., $s = 1$).
  (d) We can control how quickly a slide happens by manipulating the $\psi(\tau)$ function. A
  linear curve means the mass redistribution is taking place at a constant rate during the slide duration.
  Adjust $\beta$ to have the bulk of the redistribution happen at the front ($\beta < 1$) or closer
  to the end ($\beta > 1$).}
\label{WWKS22_Fig_slide}
\end{figure}

Depending on the shape of the seamount prior to the slide event (i.e. $h(r)$, with $h_s(r) \le h(r)$), we will
need to compute the various radii referred to in Figure \ref{WWKS22_Fig_slide}a, such as $r_1$, $r_2$, $r_c$, and $r_d$
from their corresponding heights $h_1$, $h_2$, $h_c$ and the volume of the slide. The material removed by the collapse (pink)
will be deposited at the base of the seamount (light blue) starting from a proximal height, $h_c$.  We assume this
deposit will have a linearly decaying height, enabling us to compute the distal radius $r_d$ from the volume of the slide,
$V_s$.  Clearly, $V_s$ depends on both $h(r)$ and $h_s(r)$.  We will compute this volume as $V_s = \theta (V_f - V_q)$, where $V_f$
is the fixed flank volume and whose crossection is delimited by the lines $r = r_2$, $h = 0$ and $h(r)$.  Its area $A_f$ and
centroid $\bar{r}_f$ are computed first and we then use Pappas' theorem to get the corresponding $V_f = 2\pi\bar{r}_f A_f$;
see \ref{app1} for flank areas and centroids of the different seamount shapes discussed in
Section~\ref{sec:shape}.  For $V_q$, the upper limit is $h_s(r)$ instead. We first compute the upper part of the area,
$A^u_q$ (light gray area) and later add the pedestal area $A^l_q$ below $h_1$ (darker gray area):
\begin{equation*}
A^u_q = \int_{r_2}^{r_1} \Delta h q(r) dr = \Delta h \Delta r \int_0^1 q(u) du = \Delta h \Delta r u_0 \left [ (1 + u_0) \ln \left (\frac{1 + u_0}{u_0} \right ) - 1 \right ].
\end{equation*}
To use Pappas' theorem we need the radius to the centroid, obtained via
\begin{equation*}
\bar{r}^u_q = \frac{\int_{r_2}^{r_1}q(r)rdr}{\int_{r_2}^{r_1}q(r)dr} = r_2 + \Delta r \bar{u}_q^u = r_2 + \Delta r \frac{\int_0^1q(u)udu}{\int_0^1 q(u)du}.
\end{equation*}
This ratio of integrals yields
\begin{equation*}
\bar{u}^u_q = \frac{(1 + u_0)\left [1 - u_0 \ln \left ( \frac{1+u_0}{u_0} \right ) \right ] - \frac{1}{2}}{(1 + u_0) \ln \left (\frac{1 + u_0}{u_0} \right ) - 1}.
\end{equation*}
To compute the volume of the pedestal we need its area $A^l_q = \Delta r h_1$ and centroid radius
\begin{equation*}
\bar{r}^l_q = \frac{ h_1\int_{r_2}^{r_1} rdr}{A^l_q} = \frac{1}{2} (r_1 + r_2).
\end{equation*}
The final slide volume is therefore
\begin{equation*}
V_s = \theta \left [ V_f - \left (A^u_q \bar{r}^u_q + A^l_q \bar{r}^l_q \right ) \right ],
\end{equation*}
yielding
\begin{equation}
V_s = \theta V_f - 2 \pi \theta \Delta r \left \{ \Delta h \left ( r_2 + \Delta r\bar{u}_q^u \right ) u_0 \left [ (1 + u_0) \ln \left (\frac{1 + u_0}{u_0} \right ) - 1 \right ] + \frac{h_1}{2} (r_1 + r_2) \right \}.
\label{eq:Vs}
\end{equation}
To solve for the distal radius $r_d$ we need to equate $V_s$ with the equivalent distal volume.
Given that the distal triangle thickness and area\footnote{We assume that $h(r)$ from $h_c$ to zero may be approximated
as a linear ramp.} are
\begin{equation*}
h_d(r) = h_c \left (1 - \frac{r - r_c}{r_d - r_c}\right ), \quad A_d = \frac{h_c}{2} (r_d - r_0)
\end{equation*}
we find its volume to be
\begin{equation}
V_d = 2 \pi \theta \bar{r}_d A_d = 2 \pi \theta \bar{r}_d \left [ \frac{h_c}{2} (r_d - r_0) \right ],
\label{eq:Vd}
\end{equation}
where $\bar{r}_d$ is the centroid radial distance of the distal triangle and is obtained in a similar fashion to $\bar{r}_q^u$
(except we must handle the awkward initial section separately):
\begin{equation}
\bar{r}_d = \frac{\int_{r_c}^{r_d}h_c \left (1 - \frac{r - r_c}{r_d - r_c} \right )rdr - \int_{r_c}^{r_o}h_c \left (1 - \frac{r - r_c}{r_o- r_c} \right )rdr}{A_d} = \frac{r_c + r_0 + r_d}{3}.
\label{eq:rd}
\end{equation}
Inserting (\ref{eq:rd}) into (\ref{eq:Vd}) and equating it to $V_s$ leads to the quadratic equation
\begin{equation*}
r_d^2 + r_c r_d - \left (r_0^2 + r_0 r_c + \frac{3 V_s}{\pi \theta h_c}\right ) = r_d^2 + r_c r_d - c = 0,
\end{equation*}
with unique solution
\begin{equation}
r_d = \frac{-r_c + \sqrt{r_c^2 + 4c}}{2}.
\end{equation}

\subsection{Specify the slide volume}

We may want a slide that redeposits a certain fraction $\phi_0$ of the total volume $V_0$ of the entire seamount, or
a specific volume $V_s$. In those cases we will need to compute a suitable $u_0$ given the other parameters in order to match the volumes.
Using equation (\ref{eq:Vs}) for $V_s$, we equate it to $\phi_0 V_0$ (or to the specified volume) and rearrange the equation
into a form where the right-hand side is independent of $u_0$:
\begin{equation}
\left ( r_2 + \Delta r \bar{u}_s^q \right ) u_0 \left [ (1 + u_0) \ln \left (\frac{1 + u_0}{u_0} \right ) - 1 \right ] = \frac{1}{2\Delta h} \left [\frac{V_f - V_s/\theta}{\pi \Delta r} - h_1(r_1 + r_2) \right ].
\label{eq:u0}
\end{equation}
Note we scale $V_s$ to a full 360-range to be compatible with the other volume fractions.  We then solve for
$u_0$ numerically. Also note that $V_s/\theta$ cannot exceed $V_f$ so $\phi_0$ is limited by the other parameters.
The selections for $h_1$, $h_2$ and $h_c$ limits the possible slide volume $V_s \le V_f-V^l_q$. As
$V_s \rightarrow V_f-V^l_q$, we find $u_0 \rightarrow 0$.

\subsection{Azimuthal variation in slide shape}

The above derivations assume the slide profile $h_s(r)$ only varies with radial position and is constant as a function
of the azimuth within the slide sector. Alternatively, we may wish to have the slide start at the undeformed scarp height
($h(r)$) for azimuth $\alpha_1$ and drop down to the final level ($h_s(r)$) over some azimuthal distance towards the mid-point
to give a more realistic, scalloped appearance. We model this azimuthal scaling function by
\begin{equation*}
s(\alpha) = s(\gamma) = 1 - \left |\gamma\right|^p,
\end{equation*}
which, for $p = 2$, yields a concave parabolic function. The normalized angle $\gamma$ over the range $\pm1$ is defined as
\begin{equation*}
\gamma = 2\frac{\alpha - \alpha_1}{\alpha_2 - \alpha_1} - 1.
\end{equation*}
We can use the parameter $p \ge 2$ to increase the drop-off rate from the scarp toward the middle of the section.
Now, the computed slide height will be a function of both radius and azimuth, i.e.,
\begin{equation*}
h_s(r, \alpha) = h(r) \left [1 - s(\alpha)\right ] + h_s(r) s(\alpha),
\end{equation*}
When $p \rightarrow \infty$ then $s(\alpha) \rightarrow 1$ and we recover the original radial-variation-only slide height.
For other values, see Figure~\ref{WWKS22_Fig_slide}c. For volume calculations we must now integrate $s(\alpha)$ over the azimuthal range,
which yields
\begin{equation*}
\bar{s} = \int_0^1  \gamma^p d\gamma = \frac{p}{p+1},
\end{equation*}
since $p \ge 2$. Therefore, if azimuthal variations is specified, we must scale the previously obtained slide volume from
(\ref{eq:Vs}) by $\bar{s}$ and consequently divide $V_s$ by $\bar{s}$ in (\ref{eq:u0}).

\subsection{Evolution of a slide over time}

The above results discuss the final shape of topography after a slide has occurred.  However, for time-series analysis
we will also need to look at intermediate stages.  This brings up a few further adjustments. We will assume that a
slide starts at $t_0$ and completes by $t_1$ (these are different times than those used in the construction of the seamount).
We will introduce $\psi(\tau) = \tau^\beta$ as the normalized volume
fraction of the slide as a function of normalized time $\tau = (t - t_0)/(t_1 - t_0)$.  Hence, $\psi(0) = 0$ and
$\psi(1) = 1$ and in between we require it to be a monotonically increasing curve (e.g., linear for $\beta = 1$; see
Figure~\ref{WWKS22_Fig_slide}d).  This means that, for some time $t$, the actual volume is the fraction $\psi(t)$ of
the final slide volume $V_s$. Because our radial slide function $h_s(r)$ always starts at the top scarp $h_2$) and
ends at the bottom ($h_1$) we realize that to compute a slide with a reduced volume given by $\psi(t)$ we must
recompute $u_0$ via (\ref{eq:u0}). There are two separate cases to consider:
\begin{enumerate}
  \item If a specific final landslide volume has been prescribed via $V_s = \phi_0 V_0$ then for time $t$ we simply use
  $\phi = \phi_0 \psi(t)$ when computing $u_0$.
  \item If instead the final landslide volume is \emph{not} set, but instead we provided initial parameters such as $u_0$,
  then we first need to determine the final slide volume $V_{s_0}$ from (\ref{eq:Vs}), then assign an
  equivalent volume fraction  $\phi_0 = V_s/V_0$, and finally use $\phi = \phi_0 \psi(t)$ to compute the required
  $u_0$.
\end{enumerate}
At each time step we must therefore recompute the distal radius $r_d$ but keep the start of the toe deposit at $h_c$.

\begin{figure}[h]
\centering
\includegraphics[width=12cm]{pdf/WWKS22_Fig_Nuuanu.pdf}
\caption{(a) Present relief of the Hawaiian Ridge near Oahu and the landslide deposits to the NNE.
(b) A simple model approximating Oahu after the landslide.  THe parameters used where $h_1 = 750$m, $h_2 = 6000$m,
$h_c = 300$m, $\alpha_1 = -20$, $\alpha_2 = 120$, $p = 10$ and $\phi = 20$\%.
(c) Color map shows the mass redistribution while the contour reflect the expected isostatic adjustment due for a 25 km
thick elastic plate.}
\label{WWKS22_Fig_Nuuanu}
\end{figure}
To demonstrate an application of this model we approximate the Nuuanu landslide (Figure~\ref{WWKS22_Fig_Nuuanu}),
estimated to have affected  volume of 10--40\% of the island of Oahu, using a a conic seamount with a 140 degree sector,
redistributing 20\% of the initial mass over a large area north of the island.  For our model loading an elastic plate
of 25 km thickness, the isostatic adjustment due to the large blanket of sediments peaks at 60 meters.  However, it is
the rebound beneath the main rupture area that is of more immediate interest.  For this particular model it exceeds
140 meter.  Depending on the timing of the landslides, such isostatic adjustments would very negatively affect coral
communities growing on the far side of the seamount. Note we are not presenting this as an optional model for the Nuuanu
slide but rather a case study to show the applicability of the synthetic model; yet our results are comparable to those found
by \citet{SW2000}.

\section{Density Structure of Seamounts}

\begin{figure}[h]
\centering
\includegraphics[width=9cm]{pdf/WWKS22_Fig_vp-rho.pdf}
\caption{a) Published seamount $v_p(z)$ models for 13 seamounts and oceanic islands, as cited in the text.
b) Predicted density converted from velocities using standard procedures \citep{B2005}, i.e., the Nafe-Drake relationship \citep{LND70}
for $V_p < 6$ and \citet{CM1995} for $V_p >6$. Density averages from the deep drill hole
in Hawaii \citep{JM2001} and the Azores \citep{H1979} are also shown.}
\label{WWKS22_Fig_vp-rho}
\end{figure}

For most flexural studies of seamounts it has been customary to approximate the density structure of the
volcano by a single mean load density in the modeling \citep[e.g.,~][]{WC74,WSSW06}; this also simplifies calculations.
As several large seamounts now have been imaged seismically \citep[e.g.,~][]{W1985,H1994,C1995,W1997,C1999,W1999,G2001,K2002,EMH05,C2009,W2021},
and some with a formal inversion for density structure \citep[e.g.,~][]{H1991}, it is clear that most seamounts have denser
cores than flanks.  Increasing density with depth is also generally found from samples recovered from deep boreholes
\citep[e.g.,~][]{JM2001,H1979}.
Consequently, more recent investigations into the flexural compensation beneath seamounts have considered more complex
density distributions in the calculation of flexure and gravimetric responses \citep[e.g.,~][]{C2009,W2021}. To apply
these general lessons to the majority of seamounts that have had no seismic imaging, \citet{KW2010} implemented a flexural
model with separate edifice and core densities. They found that a uniform density model would underestimate the elastic thickness
by 25\% or more compared to a variable density load.  Yet, an improved density model should reflect the information derived
from the seismically imaged seamounts.  Expanding on the efforts of \citet{M2001} and \citet{W2021}, Fig. \ref{WWKS22_Fig_vp-rho}
shows a compilation of 13 1-D seismic velocity models within a variety of seamounts and oceanic islands.  We have converted these
velocities to densities using standard conversions outlined in \citet{B2005}. Computing the median and robust median absolute deviations of the
stacked densities yielded a smooth representative curve with confidence bands.  We then obtain a weighted least squares fit for
a simple monotonic trend to the weighted median values, given by
\begin{equation*}
\rho(z) = \rho_l + \Delta \rho_s \left [ \frac{z}{H} \right ]^p,
\end{equation*}
with $\rho_l = 2260$ kg m$^{-3}$, $\Delta \rho_s = 730$ kg m$^{-3}$ and $p = 0.78$; here, $H$ was set to 7 km (dashed
horizontal line in Fig. \ref{WWKS22_Fig_vp-rho}). However, a linear model would be just as significant, with
$\rho_l = 2330$ kg m$^{-3}$, and $\Delta \rho_s = 720$ kg m$^{-3}$. Both models are shown in Fig. \ref{WWKS22_Fig_vp-rho-model}.
Clearly, the compilation from 13 diverse seamounts of variable sizes located in three oceans generally support a gradual
increase in density with depth, as would be expected from the effects of overburden and the consideration that initial
eruptions commenced at greater water depths. We note, however, that density determinations from samples in the deep Hilo
borehole shows considerable scatter with depth \citep{JM2001}, as do the shallower estimates from the Azores \citep{H1979}.
While individual seamounts may have a more heterogeneous density structure \citep[e.g.,~][]{EMH05,H1991}, for a general model
that may be applied in seamount studies in seismically (and even bathymetrically) unexplored regions we will need to impose axisymmetry.

\begin{figure}[h]
\centering
\includegraphics[width=9cm]{pdf/WWKS22_Fig_vp-rho-model.pdf}
\caption{a) Median velocity profile and median absolute deviation confidence band. b) Median density
profile and median absolute deviation confidence band, with best-fitting linear and exponential models.}
\label{WWKS22_Fig_vp-rho-model}
\end{figure}

Informed by our analysis, a general model for a representative seamount density structure is proposed as
\begin{equation}
\rho(r,z) = \rho_l + \Delta \rho_f \left [\frac{H-h(r)}{H}\right ] + \Delta \rho_s \left [ \frac{h(r)-z(r)}{H} \right ]^p
\label{eq:densmodel}
\end{equation}
where $\Delta \rho_f$ is an optional flank density increase due to water pressure, $H$ is the fixed height
of this reference seamount, $\Delta \rho_s = \rho_h - \rho_l$, with $\rho_l$ the shallowest flank density and
$\rho_h$ the highest density at the base of the reference seamount's core, while $h(r)$ is the relief of
the current seamount as a function of radial distance $r$. Finally, $z(r)$ is a point inside the seamount, and $p$ is a
positive exponent.  Because $H$ and the density parameters are fixed, smaller seamounts will have lower overall
density predictions, in agreement with observations \citep{H1991}. Note that $z$ is zero at the seamount base and increases
upwards, as do $h(r)$. As shown in Fig. \ref{WWKS22_Fig_vp-rho-model},
the power exponent $p$ may reasonably be assumed to be 1 for a linear model, but nonlinear density structures
can be explored for $p < 1$ (larger core) or $p > 1$ (smaller core). For suitable parameters, the internal (here, linear)
density structure of a model seamount is displayed in Fig. \ref{WWKS22_Fig_densitymodel}.

\begin{figure}[h]
\centering
\includegraphics[width=9cm]{pdf/WWKS22_Fig_densitymodel.pdf}
\caption{Axisymmetric density structure given by (\ref{eq:densmodel}), for $\rho_l = 2500$, $\rho_h = 3000$,
and $\Delta \rho = 200$. All densities are in kg m$^{-3}$. $H = h(0) = 6000$ m is the (untruncated) seamount height,
and $p = 1$ for a linear gradient. The guyot is conical and has a truncation $f = 0.2$.  Note the densities are
always computed from the untruncated height.}
\label{WWKS22_Fig_densitymodel}
\end{figure}

As discussed in the Section \ref{sec:shape}, seamounts grow upwards and outwards, as supported by both
geologic observations \citep[e.g.,~][]{SC2010} and dynamic arguments \cite<e.g.,>[]{LOT81}. In order to
model both flexural and gravimetric contributions accurately using established Fourier-based methodologies
\citep[e.g.,~][]{P1972,WattsBook} we need to assign a vertically-averaged density to each incremental
layer.  We integrate (\ref{eq:densmodel}) between $z_1(r)$ and $z_2(r)$ and normalize by the thickness
$\Delta z(r) = z_2(r) - z_1(r)$, yielding the vertically averaged density of a layer as a function of
radius to be
\begin{equation}
\bar{\rho}_l(r) = \rho_l + \Delta \rho_f \left [\frac{H-h(r)}{H}\right ] + \Delta \rho_s \frac{\left [h(r) - z_1(r) \right ]^{p+1} - \left [ h(r) - z_2(r) \right ]^{p+1}}{H^p (p+1)\Delta z(r)}.
\label{eq:avedenslayer}
\end{equation}
Note that if the mean radial density variation of the entire seamount is desired then setting $z_1(r) = 0$
and $z_2(r) = h(r)$ yields
\begin{equation}
\bar{\rho}_s(r) = \rho_l + \Delta \rho_f \left [\frac{H-h(r)}{H}\right ] + \frac{\Delta \rho_s}{p+1} \left [ \frac{h(r)}{H} \right ]^p.
\label{eq:avedenssmt}
\end{equation}
Finally, for comparisons with uniform density models it is helpful to know a seamount's average density as well,
which is given generally by
\begin{equation*}
\bar{\rho}_s = V^{-1}\int_V \rho(r,z)dv = 2 \pi V^{-1} \int_0^{r_0} \bar{\rho}_s(r) h(r) r dr,
\end{equation*}
and thus depends on the particular seamount shape $h(r)$ selected and its volume $V$, as found in
Section~\ref{sec:shape}. As a simple example, for an regular conic seamount with a linear density gradient
($p = 1$), no truncation ($f = 0$) and no flank densification ($\Delta \rho_f = 0$), the mean seamount density becomes
\begin{equation*}
\bar{\rho}_s = \rho_l + \frac{\Delta \rho_s h_0}{4 H},
\end{equation*}
while the more general cases are simpler to evaluate numerically.

Our morphological approximations make it simple to generate synthetic seamount models that can simulate the
growth of actual seamounts.  Furthermore, given the generalized density variation we can now assign a more realistic
density structure to such seamounts as well.  

\begin{figure}[h]
\centering
\includegraphics[width=9cm]{png/WWKS22_Fig_prisms.png}
\caption{3-D views of the internal density structure of a truncated Gaussian model using the density
relations developed herein. (top) Traditional uniform density structure, (middle) Variable axisymmetric densities
as given by (\ref{eq:densmodel}), (bottom) Simplified structure where density variations with depth
have been averaged.  While not a geologically realistic model, it allows an improved estimate of the load
to be made for use in plate flexure calculations.}
\label{WWKS22_Fig_prisms}
\end{figure}

\section{Results}

\begin{figure}[h]
\centering
\includegraphics[width=9cm]{png/WWKS22_Fig_Jasper.png}
\caption{Jasper Seamount. (a) Bathymetry range from around -4100 m to just over -600 m, (b) Variable density model
as given by the linear version of (\ref{eq:densmodel}) for $\rho_l = 2300$ kg m$^{-3}$ and $\rho_h = 2900$ kg m$^{-3}$,
(c) Calculated vertical gravity anomaly (in mGal) using the exact prism model.  Dashed line indicate cross-section
shown in Fig.~\ref{WWKS22_Fig_profile}.}
\label{WWKS22_Fig_Jasper}
\end{figure}

To demonstrate the density model developed in this paper, we will reexamine the much-studied Jasper Seamount off
the coast of California. This moderately large seamount ($\sim 3.5$ km tall) is a relatively young construction
in the Fieberling-Guadalupe seamount track \citep{B1989}, with $^{40}$Ar/$^{39}$Ar ages of 10.3-7.5 Ma, although a
small volume of later summit alkalic lava series has been dated at 4.8--4.1 Ma \citep{PSG91}. Thus, the main phase
of volcanism at this seamount is comparable in duration to that of typical Hawaiian volcanoes.  Previous gravimetric
studies have found a low bulk density of 2380 kg m$^{-3}$ with indications of a denser body on its western flank
\citep{H1991}, and of course the seismic 1-D velocity profile of Jasper \citep{H1994} was used to constrain the average
velocity-depth function in Fig~\ref{WWKS22_Fig_vp-rho}. To represent the relief of the seamount we extracted a
$\sim$ 60 by 60 meter bathymetric grid from the global multi-resolution topography synthesis \citep{R2009} and used
the -4100 m contour as the base of the volcanic feature.  Between this level and the topographic surface we created
stacks of vertical prisms with individual density contrasts determined via (\ref{eq:avedenslayer}).
We then computed both the free-air gravity and vertical gravity gradient anomalies over the set of $\sim 3$ million
prisms, and for comparison we used the relief surface and the vertically averaged and co-registered density grid
(via \ref{eq:avedenssmt}) to compute the same anomalies using spectral methods \citep{P1972}.
Figure~\ref{WWKS22_Fig_Jasper} shows the bathymetric expression of Jasper seamount, our prism model with one quadrant
removed in order to illustrate the internal density structure assigned to this seamount, and the prediction of the
vertical gravity gradient over the feature computed from the prism model.

\begin{figure}[h]
\centering
\includegraphics[width=9cm]{pdf/WWKS22_Fig_profile.pdf}
\caption{Observations and model predictions along a profile crossing Jasper seamount, where dashed lines indicate
estimated regional trends. a) Observed vertical gravity gradients from altimetry \citep{Setal2021}, with predictions
from exact line-integral and spectral methods.  There is excellent agreement between the prism and spectral method
with eight terms in the expansion \citep{P1972}. b) Same, but for free-air anomalies. c) Bathymetric relief along the
cross-section.}
\label{WWKS22_Fig_profile}
\end{figure}

To better compare predictions with observations we extract the solutions along a cross-section that extends diagonally
from the southwest to the northeast grid corner (dashed line in Fig.~\ref{WWKS22_Fig_Jasper}.  Because the observed data
also contain regional trends not captured by the model predictions we have simply adjusted the predicted curves up or down to
give an approximate fit (i.e., we assume the regional field is just a constant for this short profile).  As expected,
the observations inferred from altimetry lose some of the highest frequency components of the signal that would be
necessary to match the peak of the predicted anomaly; this is more evident for the vertical gravity gradient case where shorter
wavelengths are being amplified.  In general, we find a good fit to the observations, especially when using just the first
two order of the \citet{P1972} expansion for free-air anomalies, as recommended by \citet{MS2007}.  However, note that we
are not claiming that this particular density model is a better description of the  internal density structure of Jasper
seamount than those of previous studies.  In fact, no attempt was made to adjust the parameters in (\ref{eq:avedenslayer})
to improve the fit as this is beyond the scope of this technical brief. Nevertheless, we note that the mean density of Jasper in
our variable density model is 2370 kg m$^{-3}$, which is almost identical to what \citet{H1991} reported.

The capabilities outlined in this paper have been implemented in the Generic Mapping Tools \citep{Wetal2019}, version 6.4.
Seamount studies that require incremental building (for flexural calculations on a time-dependent rheology), including
mass redistribution following flank sector collapses, and investigations exploring non-constant density structures of larger
seamounts and their gravimetric expressions may find the tools \texttt{grdseamount}, \texttt{gravprisms} and \texttt{grdflexure}
in the {\it potential} supplement useful. While obviously simplistic, we hope the models and their implementations
in GMT will inspire scientists to use them in further studies of seamount modeling, validation and exploration.

\section{Data and code availability}

The GMT scripts used to produce all results and figures presented here
is available at \url{https://github.com/PaulWessel/SMT22}
under the BSD 3-clause open-source license.

Source code and installers for GMT are available at.
\url{https://www.generic-mapping-tools.org} under the GNU license.

%%%%%%%%%%%%%%%%%%%%%%%%%%%%%%%%%%%%%%%%%%%%%%%%%%%%%%%%%%%%%%%%%%%%%%%%%%%%%%%

\section{Acknowledgements}
This work was supported by US National Science Foundation grant xxyyyyy and a Leverhulme Visiting Professorship to P.W.
S.-S.K. acknowledges support from the National Research Foundation of Korea
(NRF-2021R1A2C1012030) and the Korea government (MOF 19992001).
This is SOEST publication no. xxxx.
